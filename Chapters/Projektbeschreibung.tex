%=========================================
% 	   Projektbeschreibung				 =
%=========================================
\chapter{Projektbeschreibung}

\section{Funktionsweise} 

Die Funktionsweise der Anwendung orientiert sich am bereits beschriebenen allgemeinen Aufbau
einer Streaming­Anwendung aus zwei Teilen, die unabhängig voneinander arbeiten und nur über
die Datenbank miteinander in Verbindung stehen.

\section{Datenmodell / Terminologie ­ Twitter}

Der Stream liefert sogenannte Status­Meldungen. Dabei handelt es sich im Grunde genommen um
eine Zusammenstellung aller Informationen zu einem bestimmten Tweet. Dazu gehören unter
anderem auch stets alle Informationen zum Autor des Tweets. Wenn es sich um einen Retweet
handelt, enthält der Status außerdem auch den Status des ursprünglichen Tweets. Da Retweets
jedoch meist nur der Verbreitung eines Tweets dienen, werden sie in der Anwendung jedoch mehr
oder weniger ignoriert und lediglich für Aktualisierungen verwendet. Somit sind die zentralen
Daten, die dem Stream entnommen und in der Datenbank gespeichert werden, Informationen zu
einem Tweet und seinem Autor.

\section{Mitgliederverwaltung}

Auch wenn das Ziel des Projekt es ist einen möglichst unbeteiligten Einblick in die Twitter-
Kommunikation zu bieten, besteht auch eine gewisse Notwendigkeit, die Nutzer längerfristig zu
erfassen, so dass sie Schlüsselwörter hinterlegen können. Das heißt, dass Benutzer sich mit einem
Benutzernamen und einem Passwort registrieren und anmelden müssen.

\section{Filter­/Suchparameter (Keywords/Prioritäten)}

Die zentralen Parameter, nach denen die Tweets durch den Stream erfasst werden und später dem
Benutzer präsentiert werden, sind dessen Schlüsselwörter. Für jeden Nutzer werden diese
Schlüsselwörter dabei mit einer Priorität (1–5) versehen, die angibt, wie wichtig dem Nutzer dieses
Schlüsselwort ist. So lassen sich die verfügbaren Tweets für einen bestimmten Benutzer immer in
einer eindeutigen Reihenfolge anzeigen, die durch weitere Filterung anhand eines Schlüsselworts
verfeinert werden kann.

\section{Zeitplanung / Skalierung}

Zu den größten Herausforderungen des Projekts gehört die Skalierung des Ausmaß der Sammlung
von Daten. Konkret geht es vor allem um die Frage, welche Daten wie lange gespeichert werden.
Es muss ein Kompromiss erzielt werden, der sowohl die praktische Durchführbarkeit als auch die
Relevanz und Verfügbarkeit der Daten für den Endnutzer gewährleistet. Da der Schwerpunkt des
Interesses auf aktuellen Ereignissen liegt, werden zunächst nur Tweets gesammelt und aufbewahrt,
die nicht bereits ein gewisses Alter erreicht haben. Das Limit wurde auf 48 Stunden festgelegt. Es
sollten sich also keine älteren Tweets im System befinden.