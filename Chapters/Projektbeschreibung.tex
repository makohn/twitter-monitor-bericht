%=========================================
% 	   Projektbeschreibung				 =
%=========================================
\chapter{Projektbeschreibung}
% Funktionsweise
Der Benutzer interagiert mit der Software über eine Web-Anwendung, die er in einem Browser startet. Im Zentrum arbeitet eine Spring-Anwendung auf einem Tomcat-Server, die die Aufgaben übernimmt die Verbindung mit Twitter herzustellen und die empfangenen Daten dem Frontend zur Verfügung zu stellen. Die Funktionsweise der Anwendung orientiert sich am bereits beschriebenen allgemeinen Aufbau einer Streaming-Anwendung aus zwei Teilen, die unabhängig voneinander arbeiten und nur über eine gemeinsame Datenbank miteinander in Verbindung stehen. Das Stream-Modul der Anwendung speichert die Tweets in der Datenbank und ein Ausgabe-Modul liefert die nötigen Daten zur Erstellung einer Webpage.
\\\\
Der Stream liefert sogenannte Status-Meldungen. Dabei handelt es sich im Grunde genommen um eine Zusammenstellung aller Informationen zu einem bestimmten Tweet. Dazu gehören unter anderem auch stets alle Informationen zum Autor des Tweets. Somit sind die zentralen Daten, die dem Stream entnommen und in der Datenbank gespeichert werden, Informationen zu einem Tweet und seinem Autor. Sowohl Tweets als auch Autoren werden von Twitter durch eindeutige IDs markiert. Diese werden auch in der Anwendung zur Identifikation verwendet. Von den sonstigen zahlreichen Daten werden für Tweets der Text, die Sprache, ein Ort, der Entstehungszeitpunkt, ein eventuell vorhandenes, angehängtes Bild\footnote{Mehrere Bilder sind zwar mögich aber unwahrscheinlich}, die Anzahl positiver Bewertungen\footnote{\textit{favoriteCount}} und die Häufigkeit der Verbreitung durch einen \textit{Retweet} aufgenommen. Zu Autoren werden Name, ein Profilbild und die Anzahl der \textit{Follower} gespeichert. Die meisten dieser Daten sind für die Ausgabe im Frontend bestimmt. Die \textit{Favorite}-, \textit{Retweet}- und \textit{Follower}-Statistiken werden zur Bewertung von Tweets in Form einer Priorisierung verwendet. \\
Für die Ausgabe von einzelnen Tweets werden alle dafür erforderlichen Informationen in einem \textit{OutputTweet} gesammelt. Dazu gehören neben den relevanten Daten zum Tweet und seinem Autor auch die aktuelle Priorität. Die Priorität eines Tweets ergibt sich sowohl aus den allgemeinen Statistiken, die sich laufend ändern, als auch aus den persönlichen Einstellungen des Benutzers, die ebenfalls ständigen Änderungen unterliegen. OutputTweets sind dadurch also immer auf bestimmte Benutzer und den aktuellen Zeitpunkt zugeschnitten.
\\\\ %Benutzerverwaltung /-service
Auch wenn es das Ziel des Projekts ist einen möglichst unbeteiligten Einblick in die Twitter-Kommunikation zu bieten, besteht auch eine gewisse Notwendigkeit, die Nutzer längerfristig zu erfassen, so dass sie Schlüsselwörter hinterlegen können. Das heißt, dass Benutzer sich mit einem Benutzernamen und einem Passwort registrieren und anmelden müssen. Für die automatische Benachrichtigung muss außerdem eine Email-Adresse angegeben werden. Auch die Benutzer werden in der Datenbank gespeichert. Die Ein- und Ausgabe von Benutzerdaten und Schlüsselwörtern erfolgt über dasselbe Modul der Spring-Anwendung, das auch für die Ausgabe der Tweets zuständig ist.
\\\\
Die zentralen Parameter, nach denen die Tweets durch den Stream erfasst und später den Benutzern präsentiert werden, sind deren Schlüsselwörter. Für jeden Nutzer werden diese Schlüsselwörter dabei mit einer Priorität von eins bis fünf versehen, die angibt, wie wichtig dem Nutzer dieses Schlüsselwort ist. Die jeweils aktuelle effektive Priorität eines Tweets ergibt sich aus allgemeinen Faktoren und der Priorisierung durch einen Benutzer. So lassen sich die verfügbaren Tweets für einen bestimmten Benutzer immer in eindeutiger Reihenfolge anzeigen. Die sehr weite Verwendung der Schlüsselwörter lässt sich am ehesten mit der Suche in Suchmaschinen und anderen sozialen Medien vergleichen, so dass viele Benutzer damit vertraut sind. Groß- und Kleinschreibung werden sowohl am Anfang als auch mitten im Wort nicht beachtet. Ebenso werden die für Twitter typischen Präfixe \glqq \#\grqq{} und \glqq @\grqq{} miteinbezogen. Sollen zwei oder mehr Suchbegriffe in einem Tweet enthalten sein, so können diese durch Leerzeichen getrennt in einem Keyword gesucht werden.\footnote{vgl. https://dev.twitter.com/streaming/overview/request-parameters\#track} \\
Neben diesen positiven Keywords soll auch die Möglichkeit bestehen ein Schlüsselwort auf eine Blacklist zu setzen. Tweets, die diese Schlüsselwörter enthalten, werden dem Benutzer nicht angezeigt. Es wird hier die genaue Schreibweise beachtet. Beide Arten von Keywords können auch pausiert werden, so dass die entsprechenden Tweets nicht mehr angezeigt werden. So wird eine Sichtung der Tweets durch den Benutzer weiterhin erleichtert, da er seine Suche zeitweilig global auf bestimmte Keywords einschränken kann oder bestimmte gesperrte Schlüsselwörter zulassen. \\
Die Sammlung von Tweets aus dem Stream wird dabei weder durch die Blacklist noch die Pausierung von Keywords beeinflusst. Beide Faktoren bestimmen lediglich, welche Tweets dem Anwender potentiell angezeigt werden. Dabei wird ihm zunächst eine übersichtliche, zahlenmäßig begrenzte Auswahl präsentiert, die sich aus den interessantesten Tweets zusammensetzt. Die Liste kann nach der Priorität der Tweets oder ihres Entstehungszeitpunkt sortiert werden. Ein weiteres mögliches Anzeigekriterium ist die Sprache, in der die Tweets verfasst werden. So kann die Darstellung etwa nur auf deutschsprachige Tweets beschränkt werden. Schließlich ist eine weitere Filterung anhand eines beliebigen Schlüsselworts möglich. Die bloße Eingabe dieses zusätzlichen Schlüsselworts führt zunächst zu einer Filterung der angezeigten Tweets, so dass nur noch diejenigen davon angezeigt werden, die auch dieses Schlüsselwort enthalten. Der Suchbegriff lässt sich jedoch auch in einer Anfrage als weiteres Schlüsselwort bei der Filterung der gesammelten Tweets in der Datenbank verwenden. Eine solche Suche zieht alle für den Nutzer gesammelten Tweets in der aktuell eingestellten Sprache in Betracht. Ein Sonderfall ist die Suche nach einem leeren Suchbegriff, die alle Tweets des Nutzer zurückgibt. Das Ergebnis lässt sich stets wie die ursprüngliche Liste durch Sortierung und Filterung weiterverarbeiten.
\\\\ %Skalierung
Zu den größten Herausforderungen des Projekts gehört die Skalierung des Ausmaßes der Sammlung von Daten. Konkret geht es vor allem um die Frage, welche Daten wie lange gespeichert werden. Es muss ein Kompromiss erzielt werden, der sowohl die praktische Durchführbarkeit als auch die Relevanz und Verfügbarkeit der Daten für den Endnutzer gewährleistet. Das größte Problem, das bei einem unskalierten Umgang mit dem Twitter-Stream auftreten kann, ist eine zu große Menge an Daten in der Datenbank, so dass Abfragen sehr lange dauern. Eine weitere Schwierigkeit dabei ist die Möglichkeit, dass durch zu viele Keywords und ein hohes Tweet-Aufkommen die Abarbeitung der Tweets durch das Stream-Modul nicht schnell genug verläuft. \\
Die Anzahl der zu einem bestimmten Zeitpunkt gespeicherten Tweets und somit der Daten in der Datenbank lässt sich durch das regelmäßige Löschen von nicht mehr aktuellen Tweets begrenzen. Außerdem werden Tweets, die ein bestimmtes Alter überschritten haben, gar nicht erst gesammelt und aufbewahrt. Das Limit wurde auf 48 Stunden festgelegt. Es sollten sich also keine älteren Tweets im System befinden. Autoren müssen nur so lange gespeichert werden wie es auch Tweets von ihnen in der Datenbank gibt. \\
Da der Schwerpunkt des Interesses auf aktuellen Ereignissen liegt, genügt eine vollständige Aufnahme der Tweets der letzten zwei Tage. Eine Alternative bzw. Erweiterung zu dieser zentralen Einschränkung ist die unvollständige Sammlung der Tweets, über die etwa anhand der allgemeinen Priorität entschieden werden kann. Weniger wichtige Tweets werden außen vor gelassen. Andererseits können allgemein besonders interessante Tweets beim Löschen übergangen werden. Solche Verfeinerungen verkomplizieren die Anwendung jedoch unnötig und verfehlen außerdem den Zweck des Projekts. Einerseits ist eine vollständige Sammlung aller Tweets zu einem Thema erstrebenswert, um eine möglichst ungefilterte Meinung präsentieren zu können. Die zur allgemeinen Bewertung eines Tweets herangezogenen Faktoren geben nur bedingt Auskunft  darüber, ob ein Tweet von Interesse ist. Andere Faktoren spielen ebenfalls eine Rolle, sind aber zu individuell um in eine allgemeine Bewertung einzufließen. Andererseits sollte die Menge an Tweets, die dem Nutzer zur Verfügung steht, immer möglichst übersichtlich und aktuell sein. \\
Neben dem Alter eines Tweets kann auch die Sprache, in der er verfasst ist, als Ausschlusskriterium dienen. Da zunächst an eine deutschsprachige Zielgruppe gedacht wird, liegt bei deutschen Tweets auch das Hauptinteresse. Außerdem werden auch Tweets in der \textit{lingua franca} des Internets Englisch gesammelt. Eine Erweiterung wäre hier leicht möglich und für Sprecher weiterer Sprachen sicherlich wünschenswert, würde jedoch in der gegenwärtigen Anwendung zu einer zu hohen Auslastung des Streams und der Datenbank führen. \\
Eine weitere Möglichkeit die Anzahl der gesammelten Tweets zu verringern ist der Ausschluss von Retweets. Wenn es sich bei einem Tweet um einen Retweet handelt, enthält das \textit{Status}-Objekt auch einen \textit{Status} des ursprünglichen Tweets. Da Retweets meist nur der Verbreitung eines Tweets dienen, werden sie in der Anwendung ignoriert und lediglich zur Aktualisierung der Statistiken des ursprünglichen Tweets verwendet. \\
Eine weitere notwendige Maßnahme ist die Beschränkung der Anzahl von Schlüsselwörtern pro Benutzer auf maximal zehn Stück. Dadurch wird einerseits sichergestellt, dass auch bei größeren Nutzerzahlen die obere Grenze von 400 Keywords eines \textit{public streams} nicht überschritten wird.\footnote{Damit ist eine maximal mögliche Nutzeranzahl von 40 Nutzern gegeben unter der Worst-Case-Bedingung, dass sich die Schlüsselwörter der Nutzer nicht überschneiden.} Andererseits begrenzt auch dies die Anzahl an vorhandenen Daten in der Datenbank und der Stream wird weniger belastet, was wiederum eine Performanzsteigerung zugunsten der Nutzererfahrung impliziert. Die Länge der Blacklist muss nicht beschränkt werden, da sie keinen Einfluss auf die Filterung des Streams und die Speicherung in der Datenbank hat, sondern nur die Ausgabe einschränkt. \\
Schließlich muss auch die Anzahl der im Frontend angezeigten Tweets kontrolliert werden. Eine zu große Anzahl an Tweets ließe sich durch den Benutzer dabei sowieso nicht überblicken. Andererseits kann die Datenbank-Anfrage und die Übertragung eines entsprechenden JSON-Dokuments bei genügend vorhandenen Tweets andernfalls mehrere Minuten dauern. In der Anwendung werden deshalb zunächst nur maximal 100 Tweets angezeigt, die nach ihrer Priorität ausgewählt werden. Sie bieten einen schnellen Überblick über die interessantesten Neuigkeiten. Diese Vorgehensweise verfälscht jedoch die Ergebnisse der Weiterverarbeitung der Tweets, also die Sortierung nach der Zeit und die Filterung nach einem Schlüsselwort, da nur noch die 100 angezeigten Tweets einbezogen werden. Deshalb besteht für den Benutzer auch immer die Möglichkeit alle für ihn gesammelten Tweets zu laden, anhand eines bestimmten Begriffs zu filtern und weiter zu bearbeiten. Dabei werden lange Ladezeiten und unübersichtlich viele Einträge in Kauf genommen, um dem Nutzer Zugriff auf alle seine Tweets bieten zu können.
%-------------------------------------------------------------------------------------------------------------------------