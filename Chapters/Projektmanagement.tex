%=========================================
% 	   Projektmanagement					 =
%=========================================
\chapter{Projektmanagement}

Als Vorgehensmodell für das Projekt wurde ein agiles Vorgehen nach dem Vorbild von Scrum und
Extreme Programming gewählt. Dies ist hauptsächlich der Tatsache geschuldet, dass die
Verwirklichung unseres Vorhabens die Einarbeitung in zahlreiche Einzelaspekte erforderte, die
dem gesamten Team neu waren. Es war nicht abzusehen, welche Teile überhaupt umsetzbar sind
und wie erfolgreich die Implementierung verlaufen wird. Deshalb sollte das Projekt zunächst nur
den eigentlichen Stream als Kern verwirklichen und dann in kleinen Iterationen um immer mehr
Funktionalität, vor allem einer Website als User Interface, erweitert werden.

\section{Teamorganisation}
Es wurde kein Projektleiter festgelegt. Entscheidungen wurden stets im Plenum getroffen. Die
Gruppe repräsentierte sich nach außen stets gemeinsam.
Die sonstige interne Organisation des Projektteams wurde ebenfalls durch die Unwägbarkeiten der
Durchführung geprägt. Da zu Beginn noch nicht klar war, welche konkreten Aufgaben es geben
wird und wie groß ihr Umfang sein wird, konnten keine klaren Zuständigkeiten verteilt werden.
Die achtköpfige Projektgruppe teilte sich jedoch schnell in mehrere kleine Teams aus zwei oder
drei Mitgliedern, die die einzelnen Aufgaben, die sich stellten, übernahmen. So stellte sich bald
eine gewisse Spezialisierung der Teams ein. Ein „Datenbank­Team“ kümmerte sich um die
Erstellung und Pflege der Datenbank. Diese Gruppe war auch für die Installation und Wartung der
neuesten Testversion auf dem Testserver im Softwarelabor zuständig. Ein „Backend­Team“ sorgte
für den Stream und ein Interface für Nutzerabfragen aus der Datenbank. Und ein „Frontend­Team“
war für die Erstellung der Website verantwortlich. Übergreifende Aufgaben wurden von einem
vierten Team übernommen. Dazu gehörten sowohl projektorganisatorische Erledigungen
(Zeiterfassung) als auch übergreifende Aspekte der Implementierung wie etwa Sicherheitsfragen.

\begin{description}
 \item [Eclipse Mars/Neon] Die Java-IDE unserer Wahl 
 \item [Maven] Ein Package/Buildmanager  
 \item [GitHub] Eine Online-Versionskontrollsystem
 \item [Trello] Ein Taskboard zum Planen von Aufgaben
 \item [Slack] Eine Online-Kommunikationsplatformm
\end{description}

