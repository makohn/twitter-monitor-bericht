%=========================================
% 	   Frontend							 =
%=========================================
\section{Frontend} 
\label{sec:frontend}
Zur Steuerung der Anwendung durch den Benutzer dient eine Website. Die einzelnen Seiten werden durch ein gemeinsames, auch in gleichen Farben gehaltenes Design, das durch CSS organisiert wird, zusammengehalten. Alle Seiten besitzen eine Kopfsektion zur Verwaltung und eine Rumpfsektion für den eigentlichen Inhalt. Auf allen Seiten, mit Ausnahme der Startseite, befindet sich im Kopfteil ein Menü zur Navigation.  Die eigentliche Funktionalität verteilt sich dabei auf die vier Seiten \textit{Tweets}, \textit{Keywords}, \textit{Einstellungen} und \textit{Anleitung}. Das Menü bietet außerdem eine \textit{Logout}-Option. \\
Die Erstellung der Webpages erfolgt über \ac{JSP}. Dabei handelt es sich um eine Programmiersprache zur dynamischen Erstellung von Webseiten durch einen Webserver, die auf HTML und Java basiert. JSPs erlauben die Einbettung von Java-Code in eine herkömmliche HTML-Webseite. In der Anwendung wird dies jedoch hauptsächlich zur Implementierung eines sicheren Logins verwendet. Ihr eigentlicher Verwendungszweck ergibt sich aus der Verwendung von Spring MVC. \acs{JSP}s stellt eine Möglichkeit dar, eine View zu implementieren. Eine Client-Anfrage wird von einem Controller der Spring-Anwendung verarbeitet. Als Antwort erstellt der Server eine Webseite anhand einer JSP und einem Datenmodell. Auf die Daten des Models lässt sich dann im JSP etwa mit der sogenannten \textit{Expression Language} zugreifen. Aber auch diese Funktionalität wird in der Anwendung kaum genutzt. Um zu häufiges Neuladen und Neukompilieren der Seiten zu verhindern, werden sie selbst durch Client-seitiges Javascript dynamisch gehalten. Eine Seite wird nur dann völlig neu geladen, wenn über das Menü auf sie gewechselt wird. Dabei werden lediglich die statischen Elemente, wie Menü und Schaltflächen, beim konkreten Seitenaufruf geladen. Die restlichen Elemente, die den eigentlichen Inhalt darstellen, werden im Nachhinein mit asynchronen Anfragen durch Javascript in JSON-Objekten nachgeladen. \\\\
%
In den folgenden Abschnitten werden die, für die unmittelbare Nutzerinteraktion relevanten Seiten hinsichtlich ihrer Funktionalität und Implementierung betrachtet. 
%
\subsection{Home}
Der Nutzer betritt die Anwendung über die Startseite. Diese erfüllt die zwei Aufgaben Login und Registrierung. Die Seite ist ebenfalls in zwei Bereiche aufgeteilt, besitzt jedoch noch kein Menü, da die Navigation zu den meisten Seiten erst im Zusammenhang mit einem bestimmten angemeldeten Nutzer sinnvoll ist. Beim Aufruf der Seite befindet sich im Kopfbereich ein Login-Interface und es wird im unteren Bereich eine Animation angezeigt, die versucht den Umstand der täglichen Flut an Tweets, den die Anwendung adressiert, auf einen Nenner zu bringen. Bereits vor der Registrierung soll dem Benutzer so ohne große Erklärung klar werden, um was es sich bei der Anwendung handelt. \\
Im Login-Teil kann sich der Anwender mit einem Benutzernamen und einem Passwort anmelden. Dabei wird neben diesen beiden Werten außerdem ein CSRF-Token übergeben, um die Login-Sitzung vor entsprechenden Täuschungsversuchen zu schützen. Die Anmeldung der Anwendung nutzt den Login-Mechanismus von Spring Security. Nach dem Login ist der verwendete Benutzername für einen Controller aus dem Request einer Seite ersichtlich. Der Nutzer wird auf die Seite \textit{Tweets} weitergeleitet, sobald seine Daten verifiziert sind. \\
Wenn noch kein Account existiert, kann der Nutzer ein neues Konto erstellen. Durch einen Klick auf die entsprechende Schaltfläche unter dem Login-Interface wird anstelle der Animation ein Registrierungsformular eingeblendet. Um die Absprungrate möglichst gering zu halten, werden vom Anwender nur wenige Daten erfragt. Außer einem Benutzernamen und einem Passwort ist lediglich die Angabe einer Email-Adresse, an die eventuell Benachrichtigungen gesendet werden sollen, erforderlich. Diese Daten müssen jedoch auch bestimmten Richtlinien entsprechen. Der Benutzername muss etwa mindestens fünf Zeichen, das Passwort mindestens acht Zeichen lang sein. Und auch die Email-Adresse muss die richtige Form aufweisen. Registrierungsanfragen mit fehlerhaften Werten werden von der Anwendung ignoriert. Deshalb wird bereits im Frontend das Format überprüft und anhand von entsprechenden farblichen Markierungen und Meldungen  auf Fehler aufmerksam gemacht. Nach der Registrierung wird die Startseite neu geladen, damit sich der Nutzer mit seinem neuen Account anmelden kann.
%
\subsection{Tweets}
Auf der Seite \textit{Tweets} kann der Nutzer die für ihn gesammelten Tweets sichten. Der Hauptbereich der Seite ist dazu in einen Anzeigebereich, der die eigentlichen Tweets auflistet, und eine Sidebar, die verschiedene Möglichkeiten zur Bearbeitung der Anzeige bietet, eingeteilt. \\
Die Liste der Tweets wird dynamisch durch eine Javascript-Funktion generiert. Zur Darstellung eines Tweets gehören neben dem eigentlichen Content, also dem Text und einem eventuellen angehängten Bild, vor allem der Name und ein Bild des Autors sowie Entstehungszeit und -ort des Tweets. Zur Übertragung der Tweet-Daten stellt die Seite mittels Javascript eine Anfrage an den Webserver. Die Tweets werden in Form eines JSON-Objekts bereitgestellt, das einer entsprechenden Funktion übergeben wird. Diese Funktion erstellt für jeden Tweet einen Container, der an die angezeigte Liste angehängt wird. Zur weiteren Bearbeitung der aktuell geladenen Tweets werden diese außerdem lokal gespeichert. Die Seite \textit{Tweets} lädt außerdem stets die Keywords des aktuellen Nutzers und hebt diese durch rote Farbe in den Tweet-Texten hervor. \\
Die Optionen in der Sidebar dienen teilweise dieser weiteren lokalen Bearbeitung von geladenen Tweets und teilweise dazu weitere Anfragen zu stellen. Die Seite lädt initial zunächst die hundert interessantesten Tweets des Benutzers. Mit dem Button \glqq Tweets aktualisieren\grqq{} wird diese anfängliche Anfrage wiederholt, um die Anzeige auf den neuesten Stand zu bringen. Die aktuell geladenen Tweets werden durch die Eingabe eines Suchbegriffs in das entsprechende Eingabefeld gefiltert. Durch einen Klick auf den Button \glqq Deep Search\grqq{} wird eine neue Anfrage gestellt, die alle Tweets des Benutzers mit diesem Suchbegriff liefert. Das Ergebnis kann weiter gefiltert werden. Die anfängliche Sortierung der Tweets nach ihrer Priorität kann auch gemäß ihres Alters erfolgen. Das Sortierkriterium wird in einem Drop-Down-Menü ausgewählt. In einem weiteren Drop-Down-Menü kann die Sprache der Tweets, die angezeigt werden, gewählt werden. \\
Eine weitere Funktionalität der Sidebar betrifft bereits die Festlegung von Keywords. Die beiden Buttons \glqq Keyword\grqq{} und \glqq Blacklist\grqq{} fügen den aktuell markierten Text in die entsprechende Liste hinzu. Dadurch kann der Nutzer bereits beim Sichten der Tweets weitere interessante Themen hinzufügen und vor allem unliebsame Themen, die sich mit interessanten Themen überschneiden, ausblenden.
%
\subsection{Keywords}
Die Seite \textit{Keywords} ermöglicht die Verwaltung von Schlüsselwörtern und der Blacklist des Benutzers. Die Keywords werden links angezeigt. Die Blacklist steht rechts. Unter jeder der beiden Listen befindet sich ein Eingabefeld und ein Button zum Hinzufügen der Eingabe in die entsprechende Liste. Außer dem Keyword selbst gibt jede Zeile der Keyword-Liste die Priorität, die der Benutzer dem Begriff zugewiesen hat, in Form von gelb ausgefüllten Sternen an. Die Begriffe auf der Blacklist müssen nicht priorisiert werden. In beiden Listen gibt es außerdem für jeden Begriff einen \glqq Pause\grqq{}-Button und einen \glqq Delete\grqq{}-Button. Wird ein Begriff auf der Keyword-Liste pausiert, so wird er in der Anzeige auf \textit{Tweets} nicht mehr in Betracht gezogen. Ein pausierter Blacklist-Begriff wird nicht mehr herausgefiltert.
%
\subsection{Einstellungen und Anleitung}
Die \textit{Einstellungen} dienen dazu die Daten des eigenen Benutzerkontos einzusehen und zu ändern. Dazu gehört vor allem die Möglichkeit sein Passwort und seine Email-Adresse zu ändern sowie die automatische Benachrichtigung von neuen Tweets zu aktivieren oder deaktivieren. Schließlich wird hier die Möglichkeit gegeben seinen Account unwiderruflich zu löschen. Bei allen Änderungen am Account wird durch einen Dialog sichergestellt, ob die Änderungen wirklich gewollt sind. \\
Obwohl darauf geachtet wurde den Zweck und die Funktionsweise der Anwendung möglichst intuitiv zu gestalten, kann nicht davon ausgegangen werden, dass jeder Nutzer die Möglichkeiten auf Anhieb erkennt. Deshalb werden auf der Seite \textit{Anleitung} außerdem auch explizite Anweisungen gegeben, wie die Bedienung ablaufen sollte. Auch hier wird jedoch auf eine möglichst kurze Erklärung abgezielt.
%--------------------------------------------------------------------------------------------------------------






