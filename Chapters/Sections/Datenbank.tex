%=========================================
% 	   Datenbank						 =
%=========================================
\section{Datenbank}

Als Datenbanksystem wird MySQL verwendet. Dies ist eine quelloffenes relationales \ac{DBMS}, welches schnell und verlässig arbeitet. Es bietet eine Menge Funktionalitäten, darunter eine Benutzerverwaltung, Unterstützung von Stored-Procedures/-Functions, Trigger, Events und eine Textindexierung. Letzteres ermöglicht ein schnelleres Mapping von Tweets und Keywords, was aufgrund der großen Datenmenge sehr nützlich ist.
\\\\
Der Name der Datenbank lautet \texttt{twitter\_monitor} und wird benutzt um Daten wie Tweets und 
Benutzerprofile des Projektes Twitter-Monitor zu speichern. Desweiteren hilft sie auch dabei 
Benachrichtigungen über interessante Tweets für die Nutzer anzulegen, welche dann über den Server 
per E-Mail versendet werden. Hilfreiche Funktionen zur Berechnung der Tweet-Priorität werden auch
von der Datenbank angeboten. 

\subsection{Stored-Procedures und Stored-Functions}

Stored-Procedures und -Functions werden direkt auf dem \acs{DBMS} ausgeführt, bieten eine hohe Performance und ermöglichen auch komplexere Operationen einfach zu implementieren. Zusätzlich wird die Kommunikation mit der Server-Anwendung verringert und damit auch der Netzwerk-Traffic. Sie erhöhen außerdem die Wartbarkeit, weil der Code nur noch zentral an einer Stelle angepasst werden muss.

\begin{table}[!ht]
\small 
\caption{Stored-Procedures (P) und Stored-Functions (F)}
  \begin{tabular}{p{3.3cm}p{5.4cm}p{3.2cm}p{1.8cm}}
    \toprule 
    \textbf{Name} & \textbf{Beschreibung} & \textbf{Input} & \textbf{Output} \\
    \hline
    \texttt{get\_personal\_prio} (F) & Errechnet die persönliche Priorität für einen Tweet passend zum Benutzer. Dafür wird die allgemeine Tweet-Prio und die vom Benutzer eingestellte Keyword-Prio mit einbezogen. & tweetId: bigint, \newline username: \newline varchar(60) & float: \newline Errechnete Priorität \\
    %------------------------------------------------------------------------------------------
    \texttt{calc\_tweet\_prio} (F) & Implementiert die Formel zur Priorisieren von Tweets & favoriteCount: int, \newline retweetCount: int, followerCount: int & float: \newline Priorität \\
    %------------------------------------------------------------------------------------------
    \texttt{get\_general\_prio} (F) & Errechnet die Priorität für einen Tweet anhand der Tweet-ID & tweetId: bigint  & float: \newline Priorität \\
    %------------------------------------------------------------------------------------------
     \texttt{user\_exists} (F) & Überprüft, ob es bereits einen Nutzer mit dieser Email-Adresse bereits gibt.  & username: \newline varchar(60) & bool: \newline 1:true/ \newline 0:false \\
    %------------------------------------------------------------------------------------------
 	 \texttt{get\_tweet\_html} (F) & Hilfsfunktion, die ein Tweet mithilfe von HTML nachbaut wie auf der Webseite & tweetId: bigint & text: \newline Tweet \newline als HTML \\
    %------------------------------------------------------------------------------------------
 	 \texttt{notify\_users} (P) & Erstellt eine Benachrichtigung zu jedem Nutzer mit den Top-Tweets aus den letzten 12 Stunden, sofern der Nutzer dies aktiviert hat & username: \newline varchar(60) & text: \newline Top-Tweets als HTML \\
    %------------------------------------------------------------------------------------------
    \texttt{get\_top\_tweets} (F) & Ermittelt zum übergebenen Nutzer die 5 besten Tweets aus den letzten 12 Stunden & &  \\
    %------------------------------------------------------------------------------------------
    \texttt{check\_preference} (F) & Überprüft ob ein Benutzer die entsprechende Einstellung gesetzt hat. Wenn ja gibt er den eingestellten Wert zurück, wenn nicht den Standard-Wert.  & preferenceType: \newline varchar(3), \newline username: \newline varchar(60) & int: \newline Wert der \newline Einstellung \\
    %------------------------------------------------------------------------------------------
	\bottomrule
  \end{tabular}
\end{table}
\newpage
Diese Funktionen kann man folgendermaßen in einem normalen \acs{SQL}-Statement 
aufrufen:

\begin{lstlisting}[style=SQL]
	select user_exists('oliverseibert'); -- Funktion 
	call notify_users(); -- Prozedur
\end{lstlisting} 
\bigskip
\textbf{Hinweis}: Den Datentyp „bool“ gibt es zwar in MySQL, es wird jedoch als Ergebnis nach Aufruf der Funktion eine 0 oder 1 zurückgegeben.


\subsection{Events}

Damit der Event-Scheduler auch nach einem Neustart läuft wurde in der Konfigurationsdatei unter 
\texttt{mysqld} der Eintrag „event\_scheduler=on“ hinzugefügt.

\begin{table}[!ht]
\caption{Events}
  \begin{tabular}{p{4.8cm}p{4.4cm}p{4.6cm}}
    \toprule 
    \textbf{Name} & \textbf{Beschreibung} & \textbf{Intervall} \\
    \hline 
    \texttt{delete\_old\_tweets} & Löscht Tweets, die älter sind als 5 Tage & Alle 720 Minuten \\
    \texttt{notify\_users\_about\_tweets} & Ruft die Prozedur notify\_users() auf & Alle 720 Minuten, immer um 08:00 Uhr und um 20:00 Uhr \\
	\bottomrule
  \end{tabular}
\end{table}

\subsection{Trigger}

\begin{table}[!ht]
\caption{Trigger}
  \begin{tabular}{p{3.6cm}p{7.6cm}p{2.6cm}}
    \toprule 
    \textbf{Name}  & \textbf{Beschreibung} & \textbf{Typ} \\
    \hline 
    \texttt{tweets\_after\_insert} & Triggert gleicht beim Einfügen eines neuen Tweets die Keywords mit dem neuen Tweet ab & after insert  \\
	\bottomrule
  \end{tabular}
\end{table}

\subsection{Benutzer}

Es wurden zwei Benutzer für die Benutzung der TwitterMonitor-Datenbank angelegt.  
\begin{description}
	\item [twitteruser: ] Wird von auf dem Server laufenden Programm Twitter-Monitor benutzt. Hat Select-Rechte auf alle Tabellen. Darf alle Stored-Procedures/Functions ausführen. Hat Update/Insert-Rechte auf die Entitäten users, authorities, keywords, tweetAuthors, tweets und tweetMedia. Jedoch beschränkt auf die Datenbank „TwitterMonitor“. 
	\item [twitteradmin: ] Wird benutzt um die TwitterMonitor-Datenbank zu administrieren, wie etwa neue Benachrichtungstypen oder neue Benutzereinstellungen anzulegen. Oder auch um die 
		Tweet-Tabelle zu leeren. \\
		Hat Select/Update/Insert/Delete-Rechte auf alle Tabellen. Darf alle Stored-
		Procedures/Functions ausführen. Jedoch beschränkt auf die Datenbank
		„TwitterMonitor“. 
\end{description}
 
\subsection{Entitäten}
 
\begin{table}[!ht]
\caption{tweets: Speicherung der Tweet-Informationen aus Twitter. }
  \begin{tabular}{p{3cm}p{5cm}p{2cm}p{1cm}p{1cm}p{1cm}}
    \toprule 
    \textbf{Attribut} & \textbf{Beschreibung} & \textbf{Datentyp} & \textbf{PK} & \textbf{FK} & \textbf{NN} \\
    \hline 
    \texttt{tweetId} & Eindeutige ID, die von Twitter übernommen wird & \texttt{bigint} & ja && ja   \\
    \texttt{autorId} &  & \texttt{bigint} & & ja & ja   \\
    \texttt{text} & Inhalt des Tweets  & \texttt{varchar(160)} & & & ja   \\
    \texttt{favoriteCount} & Anzahl der „Favourites“  & \texttt{int} & & & ja   \\
    \texttt{retweetCount} & Anzahl der „Retweets“ & \texttt{int} & & & ja   \\
    \texttt{image} & Url zu einem anhängenden Bild & \texttt{varchar(100)} & & &   \\
    \texttt{place} & Ort des Tweets & \texttt{varchar(60)} & & &   \\
    \texttt{createdAt} & Zeitpunkt, wann der Tweet auf Twitter veröffentlicht wurde & \texttt{datetime} & & & ja   \\
    \texttt{lastUpdate} & Zeitpunkt, wann der Eintrag in die Datenbank erfolgt ist & \texttt{datetime} & & &   \\
	\bottomrule
  \end{tabular}
\end{table}

\begin{table}[!ht]
\caption{tweetAuthors: Informationen zu dem Benutzer, der den Tweet veröffentlicht hat. }
  \begin{tabular}{p{3cm}p{5cm}p{2cm}p{1cm}p{1cm}p{1cm}}
    \toprule 
    \textbf{Attribut} & \textbf{Beschreibung} & \textbf{Datentyp} & \textbf{PK} & \textbf{FK} & \textbf{NN} \\
    \hline 
    \texttt{autorId} & Eindeutige ID, die von Twitter übernommen wird & \texttt{bigint} & ja && ja   \\
    \texttt{name} & Benutzername von Twitter, wird mit „@“ angesprochen & \texttt{varchar(15)} & & & ja   \\
    \texttt{screenName} & Name, wie er bei Twitter angezeigt wird  & \texttt{varchar(20)} & & &   \\
    \texttt{followerCount } & Anzahl der Benutzer, die diesem Benutzer folgen   & \texttt{int} & & &   \\
    \texttt{pictureUrl} & Url zum Profilbild & \texttt{varchar(100)} & & &   \\
    	\bottomrule
  \end{tabular}
\end{table}
      
%\begin{table}[!ht]
%\caption{users: Stellt unsere Interessenprofile da. Notwendig für die Benutzerverwaltung.}
%  \begin{tabular}{p{3cm}p{5cm}p{2cm}p{1cm}p{1cm}p{1cm}}
%    \toprule 
%    \textbf{Attribut} & \textbf{Beschreibung} & \textbf{Datentyp} & \textbf{PK} & \textbf{FK} & \textbf{NN} \\
%    \hline 
%    \texttt{username} & varchar(60) & \texttt{bigint} & ja && ja   \\
%    \texttt{email} &  E-Mail Adresse  & \texttt{varchar(60)} & & & ja   \\
%    \texttt{password} &  Verschlüsselter Password-String & \texttt{varchar(80)} & & & ja  \\
%    \texttt{enabled} & Gibt an ob das Nutzerkonto aktiv ist oder nicht   & \texttt{tinyint} & & & ja \\
%    \texttt{registeredAt} & Zeitpunkt, wann das Interessensprofil angelegt wurde & \texttt{datetime)} & & &   \\
%    	\bottomrule
%  \end{tabular}
%\end{table}

\todo[inline]{TODO: weitere Entitäten ergänzen. Wirklich an dieser Stelle ? Besser als Anhang ?}






