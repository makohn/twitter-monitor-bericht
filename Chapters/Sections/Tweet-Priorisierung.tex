%=========================================
% 	   Tweet-Priorisierung				 =
%=========================================
\section{Tweet-Priorisierung}

In diesem Abschnitt soll dargestellt werden, wie die Tweets bewertet bzw. priorisiert 
werden. Mithilfe eines selbst geschriebenen Algorithmus wird dies im Twitter-Monitor 
realisiert. Die Schwellwerte für die Berechnung der Priorität stammen aus Recherchen zu den Themen wie Twitter von den Usern benutzt wird (https://sysomos.com/inside-twitter/twitter-retweet-stats) und Erfahrungen von anderen Personen, die bereits versuchten einen Algorithmus zur Bewertung von Tweets aufzustellen (http://www.dasheroo.com/blog/whats-a-good-twitter-engagement-rate-for-your-tweets/). Mit eigenen Erfahrungswerten wird das abstrahierte Verfahren Schritt für Schritt verfeinert. Dies wird im Folgenden aufgezeigt.
\begin{description}
	\item [Ziele:] Durch eine Priorisierung der Daten sollen dem Nutzer eine selektierte Auswahl von wichtigen bzw. interessanten Tweets angezeigt werden. 
	\item [Kriterien:] zur Priorisierung der Daten aufgelistet:
	\begin{itemize}
		\item Anzahl Retweets des Tweets 
		\item Anzahl Likes des Tweets
		\item Anzahl Follower des Tweet-Autors
		\item Alter des Tweets
		\item Interesse des Benutzers
	\end{itemize}
\end{description}

\subsection*{Schritt 1: Algorithmus aufstellen}

Likes und Retweets werden ins Verhältnis zu der Anzahl Followern gestellt. Retweets sind 
unserer Meinung nach wichtiger als Likes, deswegen haben diese einen entsprechenden 
Faktor. Zuletzt wird mit 100 multipliziert, damit das Ergebnis nicht so viele 
Nachkommastellen hat und bei wichtigen Tweets etwa bei 1 liegt:

\begin{equation}
\frac{\frac{Likes}{2} + Retweets \cdot 2}{Follower} \cdot 100
\quad
\text{\textit{Priorisierung eines Tweets}}
\end{equation}
Das Ergebnis wird folgendermaßen in fünf Prioritätsstufen unterteilt (1 = höchste bis 5 = 
niedrigste):
\begin{itemize}
	\item Prio 1: >= 1
	\item Prio 2: >= 0,5
	\item Prio 3: >= 0,1
	\item Prio 4: >= 0,05
	\item Prio 5: < 0,05 
\end{itemize}
Je nach Alter  des Tweets wird dieser anders bewertet. Dies geschieht mithilfe eines  
Faktors in folgender Staffelung: 
\begin{table}[!hb]
	\centering
	\begin{tabular}{ll}
		\toprule
		\textbf{Alter} & \textbf{Faktor}  \\
		\hline
		< 1 Tag & 1 \\
		< 2 Tage & 0,9 \\
		< 3 Tage & 0,6 \\
		< 5 Tage & 0,3 \\
		> 5 Tage & 0,1 \\
		\bottomrule
	\end{tabular}
	\caption{Staffelung von Tweets nach Alter}
\end{table}
\subsection*{Auffälligkeiten}
\begin{itemize}
	\item \textbf{Problem:} Anzahl Likes und Retweets sind am Anfang nicht vorhanden. \\ \textbf{Lösung:} Tweets werden nach einer gewissen Zeit nochmals abgefragt. Zusätzlich wird, wenn ein Retweet eines vorhandenen Tweets über den Stream eingeht, die Daten des Original Tweets aktualisiert. 
	\item \textbf{Problem:} Errechnete Priorität ist mit dieser Formel nicht optimal. Beispielsweise sind Tweets mit nur 2-3 Retweets und keinem Likes und dessen Autor nur 1 Follower hat, viel zu hoch priorisiert. Bsp.: Likes = 30, Retweets = 2, Follower = 1 führt zu einer Prio von 1900. Es handelt sich hierbei oft um Spam oder Werbung. \\ \textbf{Lösung:} Die Statistik wird geglättet, indem die Anzahl Follower mit einer festen Konstante addiert werden. Eventuell kann auch die Anzahl der Follower in der Formel mithilfe einer Exponentialfunktion oder Wurzelfunktion unterstützt werden. 
	\item \textbf{Problem:} Durch Unterteilung der errechneten Priorität in verschiedene Stufen, gehen Informationen verloren \\ \textbf{Lösung:} Unterteilung in Prioritätsstufen wird nicht mehr durchgeführt.
\end{itemize}

\subsection*{Schritt 2: Algorithmus nach Erfahrungswerten verfeinern}

Von der Anzahl der Follower wird nun die Wurzel gezogen. Dies hat zur Folge, dass Twitter-
Nutzer mit wenig Followern abgeschwächt werden. Dasselbe wird auch durch die zusätzliche 
Addition mit 100 erreicht, zusätzlich ist hier eine statistische Glättung mit 
einbegriffen. Im Vergleich zur alten Formel sind die Retweets ein wenig abgeschwächt, da 
öfters Tweets aufgetaucht sind, die nur darauf abgezielt haben so viele Retweets wie 
möglich zu ernten. 

\begin{equation}
\frac{\frac{Likes}{2} + Retweets}{\sqrt{Follower} + 100} 
\quad
\text{\textit{Priorisierung eines Tweets, zweiter Versuch}}
\end{equation}

\subsection*{Schritt 3: Tweets nach Interessen des Benutzers einstufen}

Der Benutzer hat die Möglichkeit jeden seiner Keywords eine Wichtigkeit \(\kappa\) zuzuteilen. Dabei reicht 
die Skala von 1 (wenig interessant) bis 5 (sehr interessant).  Wenn ein Tweet auf mehrere vom Benutzer 
eingestellten Keywords zutrifft, werden diese addiert. Anschließend wird mit 0,2 multipliziert. Dies ist ein 
ausgedachter Wert, damit bei einer Wichtigkeit von 5 der Tweet die gleiche Priorität wie vorher hat. 

\begin{equation}
\frac{\frac{Likes}{2} + Retweets}{\sqrt{Follower} + 100} \cdot 0.2 \cdot \kappa 
\quad
\text{\textit{Individuelle Priorisierung eines Tweets}}
\end{equation}
\bigskip 
\begin{description}
	\item [Beispiel:] \hfill \\\\ Eingestellte Keywords: BVB (Wichtigkeit=5), Bundesliga (Wichtigkeit=3) \newline Tweet: „BVB empfängt zum Auftakt in der Bundesliga Mainz 05“ \newline Allgemeine Tweet-Priorität: 10 \newline Benutzer-spezifische Priorität: \(10 * ((5 + 3) * 0,2) = 16 \) 
\end{description}
Die Priorität der Tweets wird somit erst berechnet, wenn der Benutzer diese braucht und nicht mehr beim 
Eintreffen eines jeden einzelnen Tweets. Dies soll die Performance erhöhen und Last vom Server nehmen. 