%=========================================
% 	   Frontend							 =
%=========================================
\section{Frontend} 


\subsection{Anforderungen}


Dem Benutzer soll es ermöglicht werden über eine Eingabemaske ein Konto anzulegen. Das Konto dient zur 
individuellen Zuweisung von Tweets und Keywords zu einem Benutzer. Die vom Benutzer geforderten Daten 
sollten nicht zu umfangreich sein, da sich eine schnelle Abwicklung des Prozesses positiv auf die 
Absprungrate auswirkt. Trotzdem sollten die Daten den  Benutzer eindeutig identifizieren können. 
Zudem müssen Eingaben validiert werden, um Eingabefehlern vorzubeugen und Sicherheitsaspekte umsetzen 
zu können. Die Validierung sollte schon während der Laufzeit erfolgen, so dass dem Benutzer die 
Möglichkeit gegeben werden kann seine Eingaben unmittelbar zu korrigieren. \\
Es soll eine Schnittstelle bereitgestellt werden, mit dem man Keywords erzeugen, bearbeiten und 
löschen kann. Zudem muss eine einfache Vergabe von Prioritäten möglich gemacht werden. \\
Die durch ein Keyword gefilterten Tweets müssen übersichtlich dargestellt werden. Ein dargestellter 
Tweet sollte aus folgenden Teilen zusammengesetzt sein:
\begin{itemize}
\item{Tweet-Text}
\item{Tweet-Bild}
\item{Tweet-Author}
\item{Tweet-Zeit}
\end{itemize} 
Die Liste der dargestellten Tweets sollte sich  während einer Session selbstständig aktualisieren. \\
Desweiteren sollte es dem Benutzer möglich sein sich auszuloggen, um die Sicherheit von 
Mehrbenutzersystemen zu gewährleisten. Außerdem muss die Möglichkeit gegeben sein, sein Passwort zu 
ändern und gegebenenfalls eine Email-Benachrichtigung aktivieren oder deaktivieren zu können.

\subsection{Programmierspezifische Mittel}
\medskip
\subsubsection*{\acs{JSP}/HTML}
\ac{JSP} erlauben es mittels eines Präprozessors eine HTML basierte 
Website zu erstellen, die das Hinzufügen von Java-Code inmitten des HTML-Codes erlaubt. Dazu bedient 
man sich bestimmter Tag-Bibliotheken.
 
\subsubsection*{CSS}
CSS dient der Ausrichtung verschiedener Elemente auf einer Website sowie der Bestimmung farblicher 
Aspekte. Da CSS vom Browser des jeweiligen Clients interpretiert und umgesetzt wird, benötigt eine 
Änderung von CSS-Attributen keine Neukompilierung der \acs{JSP}. DIese Eigenschaft ermöglicht 
die Neuausrichtung von Elementen bei Erfüllung einer Bedingung oder durch eine Javascript-Funktion. 

\subsubsection*{Javascript}
Javascript ist eine höhere Skriptsprache die vom Browser des Clients interpretiert wird. Sie 
ermöglicht zum einen eine funktions- sowie objektorientierte Programmierung. Sie kann aktiv auf HTML-
Elemente  zugreifen und sie verändern sowie neue HTML-Elemente dynamisch erstellen. 
Javascript hat auch die Eigenschaft CSS-Datein manipulieren zu können. Da auch Javascript keine 
Neukompilierung der JSP nach sich zieht, werden Veränderungen während der Laufzeit umgesetzt.

\subsubsection*{Java} 
Mit dem Java-Framework Spring wird das Architekturmuster MVC umgesetzt. Das MVC definiert Controller, 
die als Bindeglied zwischen dem Model, welches  hauptsächlich  mit Java Umgesetzt wird und dem View 
gelten. Ein Controller kann den Datenaustausch zwischen dem Model und Browser des Clients steuern.

\subsection{Konkreter Aufbau}
Es wurden vier JSPs erzeugt und folgendermaßen umgesetzt.

\subsubsection*{Home}
Auf der Seite Home sind zwei Hauptbereiche dargestellt, die jeweils in zwei Unterbereiche 
untergliedert sind. Der erste Hauptbereich ist der \texttt{Header}. Er beinhaltet zum einen ein Logo 
und eine Login-Interface. 
Das Login-Interface beinhaltet ein Textfeld für den Benutzernamen, sowie ein besonderes 
Passworttextfeld, dass die Eingabe nicht als Klartext wiedergibt. Zum anderen enthält die Eingabemaske 
einen Button, der Mittels einer Post-Message die Daten der beiden Textfelder an einen Controller des 
Servers weiterreicht. Natürlich muss hierbei die JSP neu geladen werden. Mit der Spring-Security-
Bibliothek und einem CRF-Tokens ist es möglich diese Nachricht sicher abzufangen, an das 
Springframework weiterzureichen und eine Session für einen bestimmten Nutzer zu erstellen. Nachdem 
der Abgleich der Eingaben und der in der Datenbank gespeicherten Werte erfolgreich verlief, wird der 
Benutzer mittels eines an das  Spring-Security-Bibliothek gebundenen Controllers ,auf die Seite 
\textit{Tweets} weitergeleitet. 
Der zweite Hauptbereich enthält eine Animation die Mittels Javascript und CSS ermöglicht wurde. Dies 
geschieht indem mehrere, zeitlich versetzte Javascript-Funktionen den CSS-Code manipulieren, wobei die 
Animation selbst jedoch von CSS-Seite verwirklicht wird, da CSS auch Bewegungsabläufe darstellen kann. 
Unterhalb des Login-Interfaces befindet sich ein Hyperlink der die Darstellung der Animation 
unterbindet und eine neue Eingabemaske, die zuvor in der JSP deklariert wurde, sichtbar macht. Diese 
Eingabemaske dient dem Erstellen eines neuen Benutzerkontos. Jedes der Textfelder führt bei Eingabe 
eines Characters  eine Javascript-Funktion aus, welche die Eingaben validiert und dynamisch eine 
entsprechende Ausgabe liefert. Auch bei dieser Eingabemaske wird eine POST-Message gesendet, die aber 
nicht von Spring-Security abgefangen, sondern mittels der JSTL-Tag-Library an einen 
\texttt{Controller} gebunden wird. 

\subsubsection*{Tweets}
Während die Seite Home hauptsächlich in der JSP selbst definiert wird, nutzt die Seite Tweets vor 
allem den Vorteil von Javascript, dynamisch HTML-Elemente zu erzeugen. Wird die Tweets-Seite geöffnet, 
erstellt ein bestimmter Controller eigenständig eine JSP, dessen Inhalt alle Tweets in Form von einem 
JSON-Objektes bereitstellt. Dieses Objekt wird von einer Javaskript-Funktion aufgegriffen. Die besagte 
Funktion erstellt für jeden Tweet im Objekt einen Tweet-Container, der alle Tweet-Attribute Formatiert 
darstellt. Zudem Verfügt die Seite Tweets über eine Sidebar mit Funktionen zur Suche von Schlagwörtern 
und Sortierung innerhalb der aktuell angezeigten Liste von Tweets. Diese Funktionen wurden 
ausschließlich mit Java-Skript realisiert, um häufigen Übertragungen zwischen Client und Server 
vorzubeugen und die Performance zu verbessern.

\subsubsection*{Keywords}
Die Seite Keywords dient zur Erstellung und Darstellung von Keywords. Wie auch bei der Seite Tweets 
sorgt ein Controller zur Übergabe eines JSON-Objektes. Diesmal werden keine Tweets sondern Keywords 
übergeben. Die durch Javascript erzeugten Keyworter bestehen aus drei Elementen. Dem Keyowrd selbst , 
fünf Sternen zur Symbolisierung  der Priorität und einem Symbol zum Löschen des Kewords. Die Sterne 
wurden mit einem Hover-Effekt belegt, der die Anzahl der Prioritätspunkte wiedergibt. Bei einem Klick 
auf einen Stern wird mit einem Javascript-Framework namens AJAX eine Nachricht an den Server gesendet 
ohne dabei die Seite neu zu Laden. Diese Nachricht wird von einem Controller empfangen und speichert 
die Priorität eines Keywords in der Datenbank. Um ein neues Keyword zu erstellen wurde unter der Liste 
von bereits existierenden Keywords ein Textfeld mit einem Button implementiert. Auch hier gilt das 
AJAX-Prinzip, um die Seite nicht neu laden zu müssen.
